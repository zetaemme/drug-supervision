\documentclass[a4paper]{article}
\usepackage[italian]{babel}
\usepackage[utf8x]{inputenc}
\usepackage{graphicx}

\topmargin = 0.45in
\evensidemargin = 0in
\oddsidemargin = 0in
\textwidth = 6.25in
\textheight = 9in
\headsep = 0.25in

\begin{document}
	\clearpage
	
	\begin{titlepage}
		\centering
		\vspace*{\fill}
		
		\includegraphics[width=0.15\textwidth]{logo.png}\par\vspace{1cm}
		{\scshape\LARGE Universita' degli Studi di Verona \par}
		\vspace{1cm}
		
		{\scshape\Large Ingegneria del Software \par}
		\vspace{1.5cm}
		\line(1, 0){250}
		
		{\Large\bfseries Documentazione al prototipo \par}
		\line(1, 0){250}
		
		{\Large\itshape Andrea Soglieri, Mattia Zorzan \par}
		\vspace{5cm}
		\vspace*{\fill}
		
		{\Large \today\par}
	\end{titlepage}
	\thispagestyle{empty}
	\newpage
	\tableofcontents
	\newpage
	
	\section{Introduzione}
		L'obiettivo dell'elaborato e' lo sviluppo di un prototipo software per la gestione delle terapie farmacologiche da parte dei medici di base.
		
		In questa relazione ci proponiamo di raccolgiere la documentazione sviluppata e di fornire spiegazioni dettagliate sulle scelte progettuali ed implementative.
	\section{Requisiti}
		Di seguito viene riportata la consegna dell'elaborato.
		
		\begin{changemargin}{1cm}{0.5cm}
			Si vuole progettare un sistema software per gestire le segnalazioni di reazioni avverse (ad esempio, asma, dermatiti, insufficienza renale, ...) da farmaci. Ogni segnalazione e' caratterizzata da un codice univoco, dall'indicazione del paziente a cui fa riferimento, dall'indicazione della reazione avversa, dalla data della reazione avversa, dalla data di segnalazione, e dalle terapie farmacologiche in atto al momento della reazione avversa. Per ogni paziente sono memorizzati: un codice univoco, l'anno di nascita, la provincia di residenza e la professione. Per ogni paziente e' possibile memorizzare gli eventuali fattori di rischio presenti (paziente fumatore, iperteso, sovrappeso, ...), anche piu' d'uno. Ogni fattore di rischio e' caratterizzato da un nome univoco, unadescrizione e il livello di rischio associato. Ogni terapia farmacologica e' caratterizzata da: paziente a cui si riferisce, segnalazioni a cui e' legata, farmaco somministrato, dose, frequenza giornaliera, data di inizio e data di fine della terapia stessa. Per ogni reazione avversa sono memorizzati un nome univoco, un livello di gravita' (da 1 a 5) e una descrizione generale, espressa in linguaggio naturale. Una reazione avversa puo' essere legata a molte segnalazioni. Per ogni paziente sono memorizzati per ogni anno il numero di reazioni avverse segnalate ed il numero di terapie farmacologiche relative. Il sistema deve supportare i medici che effettuano la segnalazione. Dopo opportuna autenticazione, il medico viene introdotto ad una interfaccia che permette l'inserimento dei dati delle reazioni avverse e dei pazienti. Il codice univoco dei pazienti e' gestito dal sistema, che tiene traccia dei pazienti indicati da ogni medico. Ogni medico vede solo i codici identificativi dei pazienti, dei quali ha gia' segnalato qualche reazione avversa. Ad ogni fine settimana o quando il numero di segnalazioni raggiunge la soglia di 50, il sistema manda un avviso ad uno dei farmacologi responsabili della gestione delle segnalazioni di reazioni avverse. Il farmacologo, dopo autenticazione, accede alle segnalazioni e puo' effettuare alcune analisi di base (quante segnalazioni per farmaco, quante segnalazioni gravi in settimana, ...). Il sistema, inoltre, avvisa il farmacologo quando un farmaco ha accumulato nell'anno oltre 10 segnalazioni di gravita' superiore a 3. In base alle segnalazioni e agli avvisi del sistema, il farmacologo puo' proporre di ritirare il farmaco dal commercio immediatamente, di attivare una fase di controllo del farmaco, di mettere il farmaco fra quelli che richiedono un monitoraggio piu' attento. Tali proposte vengono registrate dal sistema, che tiene traccia di tutte le proposte relative ai farmaci segnalati.
		\end{changemargin}
	
	\section{UML}
		In questa sezione sono presentati i diagrammi richiesti per la documentazione del prototipo.
		
		
\end{document}