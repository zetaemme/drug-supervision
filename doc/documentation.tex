\documentclass[a4paper, 11pt]{article}
\usepackage[italian]{babel}
\usepackage[utf8x]{inputenc}
\usepackage[T1]{fontenc}
\usepackage{graphicx}
\usepackage{longtable}
\usepackage{float}
\usepackage{wrapfig}
\usepackage{rotating}
\usepackage[normalem]{ulem}
\usepackage{amsmath}
\usepackage{calc}
\usepackage{lastpage} % Required to determine the last page for the footer
\usepackage{extramarks} % Required for headers and footers
\usepackage{textcomp}
\usepackage{marvosym}
\usepackage{wasysym}
\usepackage{amssymb}
\usepackage{hyperref}
\usepackage{enumitem}
\usepackage[a4paper]{geometry}
\usepackage[section]{placeins}
\usepackage{lmodern}
%\usepackage[scaled=.85]{beramono}
\usepackage{array}
\usepackage{subcaption}
\usepackage{color}
\usepackage[usenames,dvipsnames,svgnames,table]{xcolor}
\usepackage{hyperref}
\hypersetup{
	colorlinks=true,
	linkcolor=black,
	urlcolor  = blue,
}
%\geometry{hmargin=2.3cm}

\topmargin=-0.45in
\evensidemargin=0in
\oddsidemargin=0in
\textwidth=6.25in
\textheight=9in
\headsep=0.25in

\usepackage[normalem]{ulem} % [normalem] prevents the package from changing the default behavior of `\emph` to underline.
\usepackage{fancyhdr}
\usepackage{array}

\renewcommand{\baselinestretch}{1.1} 

\newcommand{\hmwkTitle}{Drug Supervision}

\pagestyle{fancy}
\lhead{\nouppercase{\leftmark}}
\rhead{\nouppercase{\rightmark}}
\chead{\hmwkTitle}
\lfoot{}
\cfoot{\thepage}
\rfoot{}
\renewcommand{\headrulewidth}{0.4pt}
\renewcommand{\footrulewidth}{0.4pt}


\def\changemargin#1#2{\list{}{\rightmargin#2\leftmargin#1}\item[]}
\let\endchangemargin=\endlist 

\newcolumntype{L}{>{\fontseries{l}\selectfont\arraybackslash}p{0.7\textwidth}}

\newcommand{\safeImage} [2] [\textwidth]{%
	\begin{figure}[H]
		\centering
		\IfFileExists{#2}{\includegraphics[width={#1},keepaspectratio]{#2}}{\includegraphics{example-image-a}}
	\end{figure}
}

\newcommand{\breakcell}[2][l]{\begin{tabular}[t]{@{}#1@{}}#2\end{tabular}}

\newcommand{\usecase}[5]{%
	\begin{table}[H]
		%\vspace*{0.5 cm}
		\centering
		\renewcommand{\familydefault}{\ttdefault}\normalfont
		\begin{tabular}{| >{\fontseries{b}\selectfont} p{0.3\textwidth} | L |}
			\hline
			ID&{#1}\\\hline
			Attori&{#2}\\\hline
			Precondizioni&{#3}\\\hline
			Sequenza&{#4}\\\hline
			Postcondizioni&{#5}\\\hline
		\end{tabular}
	\end{table}
	
}

\begin{document}
	\clearpage
	
	\begin{titlepage}
		\centering
		\vspace*{\fill}
		
		\includegraphics[width=0.15\textwidth]{logo.png}\par\vspace{1cm}
		{\scshape\LARGE Universita' degli Studi di Verona \par}
		\vspace{1cm}
		
		{\scshape\Large Ingegneria del Software \par}
		\vspace{1.5cm}
		\line(1, 0){250}
		
		{\huge\bfseries Drug Supervision\par}
		{\Large\bfseries Documentazione al prototipo \par}
		\line(1, 0){250}
		
		{\Large\itshape Andrea Soglieri, Mattia Zorzan \par}
		\vspace{5cm}
		\vspace*{\fill}
		
		{\Large \today\par}
	\end{titlepage}

	\newpage
	\tableofcontents
	\newpage
	
	\section{Introduzione}
		L'obiettivo dell'elaborato e' lo sviluppo di un prototipo software per la gestione delle terapie farmacologiche da parte dei medici di base e segnalare eventuali problemi relativi ai farmaci.
		
		In questa relazione ci proponiamo di raccolgiere la documentazione sviluppata e di fornire spiegazioni dettagliate sulle scelte progettuali ed implementative.
	\section{Requisiti}
		Di seguito viene riportata la consegna dell'elaborato.	
	\begin{changemargin}{1cm}{0.5cm}
		\textit{Si vuole progettare un sistema software per gestire le segnalazioni di reazioni avverse (ad esempio, asma, dermatiti, insufficienza renale, ...) da farmaci.\newline Ogni segnalazione e' caratterizzata da un codice univoco, dall'indicazione del paziente a cui fa riferimento, dall'indicazione della reazione avversa, dalla data della reazione avversa, dalla data di segnalazione, e dalle terapie farmacologiche in atto al momento della reazione avversa.\newline Per ogni paziente sono memorizzati: un codice univoco, l'anno di nascita, la provincia di residenza e la professione. Per ogni paziente e' possibile memorizzare gli eventuali fattori di rischio presenti (paziente fumatore, iperteso, sovrappeso, ...), anche piu' d'uno. Ogni fattore di rischio e' caratterizzato da un nome univoco, unadescrizione e il livello di rischio associato.\newline Ogni terapia farmacologica e' caratterizzata da: paziente a cui si riferisce, segnalazioni a cui e' legata, farmaco somministrato, dose, frequenza giornaliera, data di inizio e data di fine della terapia stessa. Per ogni reazione avversa sono memorizzati un nome univoco, un livello di gravita' (da 1 a 5) e una descrizione generale, espressa in linguaggio naturale. Una reazione avversa puo' essere legata a molte segnalazioni. Per ogni paziente sono memorizzati per ogni anno il numero di reazioni avverse segnalate ed il numero di terapie farmacologiche relative.\newline Il sistema deve supportare i medici che effettuano la segnalazione. Dopo opportuna autenticazione, il medico viene introdotto ad una interfaccia che permette l'inserimento dei dati delle reazioni avverse e dei pazienti. Il codice univoco dei pazienti e' gestito dal sistema, che tiene traccia dei pazienti indicati da ogni medico. Ogni medico vede solo i codici identificativi dei pazienti, dei quali ha gia' segnalato qualche reazione avversa.\newline Ad ogni fine settimana o quando il numero di segnalazioni raggiunge la soglia di 50, il sistema manda un avviso ad uno dei farmacologi responsabili della gestione delle segnalazioni di reazioni avverse. Il farmacologo, dopo autenticazione, accede alle segnalazioni e puo' effettuare alcune analisi di base (quante segnalazioni per farmaco, quante segnalazioni gravi in settimana, ...). Il sistema, inoltre, avvisa il farmacologo quando un farmaco ha accumulato nell'anno oltre 10 segnalazioni di gravita' superiore a 3.\newline In base alle segnalazioni e agli avvisi del sistema, il farmacologo puo' proporre di ritirare il farmaco dal commercio immediatamente, di attivare una fase di controllo del farmaco, di mettere il farmaco fra quelli che richiedono un monitoraggio piu' attento. Tali proposte vengono registrate dal sistema, che tiene traccia di tutte le proposte relative ai farmaci segnalati.}
	\end{changemargin}
	
	\section{UML}
		In questa sezione sono presentati i diagrammi richiesti per la documentazione del prototipo.
		
		\subsection{Use Cases}
			Gli use case sono stati suddivisi in due macroaree definite dalla tipologia di utente che effettua il login.\newline Esse possono essere suddivise in base alla GUI:
			
			\begin{enumerate}[nosep]
				\item Vista del medico di base.
				\item Vista del farmacologo.
			\end{enumerate}
		
		\safeImage[0.7\textwidth]{Use-cases.png}
		
		Di seguito alcune schede che descrittive degli use case sopracitati.
		
		% Creare immagini per usecases
		
		\usecase{UC1: Inserimento paziente.}
		{Medico}
		{Il medico deve aver effettuato il login.}
		{%
			\begin{enumerate}[label*=\arabic*., nosep]
				\item Il medico apre il menù "File", sposta il cursore sopra la voce "New" scegliendo la voce "Patient". Viene visualizzato il dialog corrispondente.
				\item Inserisce i dati relativi alla persone e gli eventuali Risk Factor.
				\item Se il medico preme " New" in corrispondenza della riga "Risk Factor":
				\begin{enumerate}[label*=\arabic*., nosep]
					\item Inserisce i dati relativi al nuovo Risk Factor.
					\item Se il medico preme "Add":
					\begin{enumerate}[label*=\arabic*., nosep]
						\item Viene effettuata la validazione dei dati immessi. Se sono validi:	
						\begin{enumerate}[label*=\arabic*., nosep]
							\item Il Risk Factor viene aggiunto al Database.
							\item Il dialog viene chiuso.
						\end{enumerate}
					\end{enumerate}
				\end{enumerate}
				\item Se il medico preme "Add":				
				\begin{enumerate}[label*=\arabic*., nosep]
					\item Viene effettuata la validazione dei dati immessi. Se sono validi:	
					\begin{enumerate}[label*=\arabic*., nosep]
						\item Il paziente viene aggiunto al Database.
						\item Il dialog viene chiuso.
					\end{enumerate}
				\end{enumerate}
			\end{enumerate}
		}
		{Nuovo paziente aggiunto correttamente.}
		
		\usecase{UC2: Inserimento report.}
		{Medico}
		{Il medico deve aver effettuato il login.}
		{%
			\begin{enumerate}[label*=\arabic*., nosep]
				\item Il medico apre il menù "File", sposta il cursore sopra la voce "New" scegliendo la voce "Report". Viene visualizzato il dialog corrispondente.
				\item Inserisce i dati relativi al report, alla reaction e alla terapia farmacologica.
				\item Se il medico preme "New" in corrispondenza della riga Reaction: 
				\begin{enumerate}[label*=\arabic*., nosep]
					\item Inserisce i dati relativi alla nuova reaction.
					\item Se il medico preme "Add":
					\begin{enumerate}[label*=\arabic*., nosep]
						\item Viene effettuata la validazione dei dati immessi. Se sono validi:	
						\begin{enumerate}[label*=\arabic*., nosep]
							\item La reaction viene aggiunta al Database.
							\item Il dialog viene chiuso.
						\end{enumerate}
					\end{enumerate}
				\end{enumerate}
				\item Se il medico preme "New" in corrispondenza della riga Therapy: 
				\begin{enumerate}[label*=\arabic*., nosep]
					\item Inserisce i dati relativi alla nuova terapia farmacologica.
					\item Se il medico premo "New" in corrispondenza della riga Drug:
					\begin{enumerate}[label*=\arabic*., nosep]
						\item Inserisce il nome del nuovo farmaco.
						\item Se il medico preme "Add":
						\begin{enumerate}[label*=\arabic*., nosep]
							\item Viene effettuata la validazione dei dati immessi. Se sono validi:	
							\begin{enumerate}[label*=\arabic*., nosep]
								\item Il farmaco viene aggiunto al Database.
								\item Il dialog viene chiuso.
							\end{enumerate}
						\end{enumerate}
					\end{enumerate}
					\item Se il medico preme "Add":
					\begin{enumerate}[label*=\arabic*., nosep]
						\item Viene effettuata la validazione dei dati immessi. Se sono validi:	
						\begin{enumerate}[label*=\arabic*., nosep]
							\item La terapia farmacologica viene aggiunta al Database.
							\item Il dialog viene chiuso.
						\end{enumerate}
					\end{enumerate}
				\end{enumerate}
				\item Se il medico preme "Add":				
				\begin{enumerate}[label*=\arabic*., nosep]
					\item Viene effettuata la validazione dei dati immessi. Se sono validi:	
					\begin{enumerate}[label*=\arabic*., nosep]
						\item Il report viene aggiunto al Database.
						\item Il dialog viene chiuso.
					\end{enumerate}
				\end{enumerate}
			\end{enumerate}
		}
		{Nuovo report aggiunto correttamente}
		
		\usecase{UC3: Cancella paziente.}
		{Medico}
		{Il medico deve aver effettuato il login.}
		{%
			\begin{enumerate}[label*=\arabic*., nosep]
				\item Il medico apre il menù "File" e sceglie la voce "Delete"
				\item Viene mostrato un dialog che chiede di confermare la scelta.
				\item Se il medico preme "Ok":
				\begin{enumerate}[label*=\arabic*., nosep]
					\item Il paziente viene eliminato dal Database.
					\item Il dialog viene chiuso.
				\end{enumerate}					
			\end{enumerate}
		}
		{Paziente eliminato correttamente.}
		
		\usecase{UC4: Visualizzare report relativi ad un certo farmaco.}
		{Farmacologo}
		{Il farmacologo deve aver effettuato il login}
		{%
			\begin{enumerate}[label*=\arabic*., nosep]
				\item Il farmacologo può scegliere un determinato farmaco attraverso la ChoiceBox.
				\item Vengono mostrati i report relativi a quel farmaco nella TableView
				\item Se il farmacologo sceglie "Remove":
				\begin{enumerate}[label*=\arabic*., nosep]
					\item Viene settato a "true"  l'opzione di rimozione nel Database.
				\end{enumerate}
			\item Se il farmacologo sceglie "Inspect":
			\begin{enumerate}[label*=\arabic*., nosep]
				\item Viene settato a "true"  l'opzione d'ispezione nel Database.
			\end{enumerate}
			\item Se il farmacologo sceglie "Close Monitor":
			\begin{enumerate}[label*=\arabic*., nosep]
				\item Viene settato a "true"  l'opzione di chiusura del monitoraggio nel Database.
			\end{enumerate}
			\end{enumerate}
		}
		{I farmaci possono essere rimossi, ispezionati o ne può essere chiuso il monitoraggio.}
		\newpage
		
		\subsection{Activity Diagram}
		Gli activity diagram di seguito servono a esplicitare meglio il flusso d'esecuzione degli use case descritti in precedenza.
		
		\safeImage[0.6\textwidth]{Inserisci_Paziente.png}
		\safeImage[0.6\textwidth]{Inserisci_Report.png}
		
		\newpage
		Come su può vedere nella corrispondente tabella, lo UC2 incorpora l'aggiunta di una nova terapia che a sua volta contiente la possibilità di aggiungere un nuovo farmaco e  l'aggiunta di una nuova reazione. Nonostante siano parte dello stesso Use Case abbiamo deciso di dargli degli Activity Diagram.
		
		\safeImage[0.6\textwidth]{Inserisci_Terapia.png}
		\safeImage[0.6\textwidth]{Inserisci_Farmaco.png}
		\safeImage[0.6\textwidth]{Inserisci_Reazione.png}

		\subsection{Class Diagram}
		\safeImage[0.6\textwidth]{Package_Model.png}
		\safeImage[0.6\textwidth]{Package_Utils.png}
		\safeImage[\textwidth]{View-Controller_Class_Diagram.png}
		
	\section{Scelte Progettuali}
		\subsection{Sviluppo}
		Abbiamo scelto di sviluppare il prototipo utilizzando il linguaggio Java vista la sua grande flessibilità e la sua estesa diffusione in quello che è il mercato odierno. Nonostante questo, vista la nostra voglia di ampliare le nostre conoscenze del linguaggio al di fuori degli argomenti trattati a lezione, abbiamo deciso di utilizzare la libreria grafica JavaFX invece di Swing consapevoli del fatto che non avremmo potuto far uso di un designer di interfacce grafiche.\newline
		La libreria supporta completamente gli stili CSS, da noi poco utilizzati in quanto soddisfatti del risultato ottenuto con lo stile standard dei componenti.
		
		\subsection{Metodologia di Sviluppo}
		Il progetto è stato interamente sviluppato utilizzando Git come sistema di code versioning e GitHub per l'hosting del repository. Abbiamo seguito la metodologia di sviluppo Agile: il lavoro è stato diviso in due macroaree (area logica e GUI).\newline 
		A sua volta ogni area è stata divisa in task più piccoli in modo da alleggerire il carico di lavoro e testati man mano che venivano inseriti nel codice. I design pattern sono stati scelti nella fase di sviluppo in base a quelle che erano le nostre necessità al fine di ottenere un codice performante ma allo stesso tempo leggibile.
		
		\subsection{Database}
		Nel corso dello sviluppo del prototipo si è manifestata la necessità di mantenere i dati nel tempo, per questo abbiamo implementato un basilare Database appoggiandoci ad SQLite3. Purtroppo al momento dello sviluppo le nostre competenze in basi di dati si limitavano a quanto appreso durante gli anni di liceo, per questo più che essere utilizzato come un vero e proprio RDBMS  è stato implementato quasi come se i dati fossero organizzati in file.\newline
		È stata creata una classe DBConnection per aprire e chiudere la connessione con il database e per ogni classe nel Model è stata creata un'apposita classe DAO al fine di accedere alla base di dati e creare liste che rappresentassero l'intero recordset di una determinata table.
		 
		\subsection{Organizzazione della GUI}
		È possibile trovare tre sezioni principali della GUI del prototipo:
		\begin{enumerate}[label*=(\alph*)., nosep]
			\item Login: permette l'autenticazione del medico o del farmacologo, prerequisito per ogni Use-Case.
			\item View Medico: il focus è sui pazienti legati al medico autenticato. In questa view è possibile aggiungere, eliminare o visualizzare i dati relativi ad un paziente.
			\item View Farmacologo: il focus è sui report relativi alle terapie farmacologiche. Il farmacologo autenticato avrà la possibilità di monitorare gli effetti di un farmaco e, nel caso fosse dannoso, potrà rimuoverlo, ispezionarlo o chiuderne il monitoraggio. 
		\end{enumerate}
	
		\begin{figure}[H]
			\begin{subfigure}[h]{\textwidth}
				\centering
				\includegraphics[height=0.3\textheight]{Login-view.png}
				\caption{Schermata di Login}
			\end{subfigure}
		
			\begin{subfigure}[h]{\textwidth}
				\centering
				\includegraphics[height=0.3\textheight]{Medic-view.png}
				\caption{Vista del Medico}
			\end{subfigure}
		
			\begin{subfigure}[h]{\textwidth}
				\centering
				\includegraphics[height=0.3\textheight]{Pharmacologist-view.png}
				\caption{Vista del Farmacologo}
			\end{subfigure}
		\end{figure}
	
		\subsection{MVC Pattern}
		L'applicazione del pattern prevede la suddivisione logica delle classi in tre componenti: Model, View e Controller. \newline
		
		\safeImage[0.6\textwidth]{MVC.png}
		
		\subsection{Singleton Pattern}
		Per validare l'istanza dell'applicazione, secondo la nostra visione dell'elaborato, il pattern che meglio si addiceva alle nostre esigenze è il Singleton. Esso è stato utilizzato nella classe \textit{LoginController}. Tramite un istanza da noi chiamata \textit{loginInstance}, che memorizza lo username come stringa, permette di verificare l'avvenuto login tramite un metodo \textit{private} chiamato \textit{isLogged}. Questo restituisce un booleano, \textit{true} se loginInstance  ha un valore diverso da \textit{null}, \textit{false} altrimenti.    
		
		\subsection{Observer Pattern}
		All'interno del codice non è stato implementato il modello teorico dell'Observer Pattern. Nonostante questo, il rapporto \textit{Listener-Controller} esegue lo stesso tipo di operazioni. Nella classe \textit{MainPage} viene creato un Listener che "ascolta" le variazioni all'inerno della TabelView, se ne rileva una provvede a modificare le Label all'interno della GridPane. Anche se non implementato direttamente da noi, l'uso fatto del Listener corrisponde a quello del metodo \textit{update()} nello schema teorico dell'Observer Pattern.

		\subsection{Data Access Object Pattern}
		Per la gestione dei dati a database abbiamo ritenuto opportuno sfruttare il Data Access Object Pattern. Questo ci ha permesso di trattare ogni dato come fosse nativo di Java, senza dover passare a conversioni all'interno dei Controller. Il tutto è stato gestito tramite interfacce chiamate \textit{DAO} e implementazioni denominate \textit{DAOImpl}, i cui metodi definiscono oggetti ottenuti tramite l'uso di basiche query SQL.
		
		\section{Validazione}
		Per la validazione dei dati inseriti ci siamo appoggiati al sistema di Alert fornito da JavaFX. Vengono effettuati controlli tramite \textit{if statements} e \textit{try/catch} al fine di poter garantire una corretta esecuzione.\newline
		Per quanto riguarda la GUI non ci è stato possibile testarla automaticamente, abbiamo quindi provveduto ad eseguire un test manuale delle funzionalità. Ci siamo inoltre appoggiati a colleghi ed amici al fine di raccogliere un feedback veritiero sulle prestazioni del prototipo.
		
		\section{Conclusioni}
		Obiettivo del progetto non era quello di sviluppare un prodotto finito e completo ma di realizzare un prototipo rispettando le linee guida. Per questo non sono state implementate alcune funzionalità e l'interfaccia è rimasta ``grezza''. 
		l progetto completo può essere trovato alla pagina di \href{https://github.com/zetaemme/drug-supervision}{Github}. 
		
\end{document}